\documentclass{article}
\usepackage[utf8]{inputenc}
\documentclass{article}

\title{Trabalho de Modelagem de Sistemas Dinâmicos - ENGC35}
\author{Jesse de Oliveira, Henrique Poleselo e Luis Gustavo Christensen}
\date{Universidade Federal da Bahia - 2019.2}

\begin{document}

\maketitle

\section{Introdução}
O intuito do trabalho é explorar a modelagem de sistemas dinâmicos, devido ao interesse na área de robótica por todos membros da equipe e pela crescente demanda na industria pela automatização de processos, inclusive com robótica...
Escolhemos alguma aplicação que pudesse integrar o controle PID, assunto visto na disciplina de ENGC35.

\section{Objetivos}
Usar controle com visão computacional... Testar diferentes métodos de modelagem para o nosso sistema. Uma aplicação que não possui um ponto de partida definido..

\section{Descrição do Problema}
Apresentar o protótipo, lista dos materiais usados
mostrar a precisão da câmera, falar que a câmera é nosso sensor

Materiais utilizados:
\begin{itemize}
  \item Base de acrílico 6mm de espessura
  \item Módulo ponte H dupla L298N
  \item Dois (2) motores DC
  \item Duas rodas de 68mm
  \item Raspberry Pi
  \
\end{itemize}


\section{Modelagem}

\section{Projeto do Controlador}
Analisar que tipo de controlador iremos usar, se é PI, PID.. para controlar a planta em questão, precisamos descrever os metodos de sintonia do controlador

\section{Resultados Experimentais}

\section{Conclusão}

\section{Referências Bibliográficas/Fontes}
\href{https://www.element14.com/community/community/raspberry-pi/raspberrypi_projects/blog/2019/01/04/automation-pid-based-dc-motor-controller-using-the-raspberry-pi}

\href{https://www.youtube.com/watch?v=W7cV9_W12sM}

\href{https://medium.com/@Keithweaver_/controlling-dc-motors-using-python-with-a-raspberry-pi-40-pin-f6fa891dc3d}

\end{document}
